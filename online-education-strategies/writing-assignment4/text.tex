\documentclass[paper=a4,12pt]{article}

\usepackage[american]{babel}
\usepackage{csquotes}
\usepackage[style=apa]{biblatex}
\usepackage{fontspec}
\usepackage[margin=1in]{geometry}
\usepackage{setspace} \doublespacing
\setmainfont{times new roman}
\addbibresource{export.bib}
\begin{document}
In the case study, Narisa works for a company called K9 Dura-Corp, which manufactures pet supplies in Michigan. Her current mission as a procurement specialist is to find an alternative fabric lining for their dog cot because their old manufacturer had gone out of business. She has found two possible suppliers but doesn't know which to choose from. The first candidate, Fab-Right Inc, sells cheaper and more durable fabric. They, however, exploit their labor and, according to what she believes, may be hiding something about the environmental impact of their manufacturing process. The second candidate, All-Fabric Inc, sells less durable fabric. Also, they are criticized by residents for contaminating local water. Since their sales are going down, K9 Dura-Corp cannot tolerate production delays. If there's a delay, they may have to lay off some of the workers, probably including Narisa.

I used a critical thinking strategy called "PrOACT" to analyze this case. This is a framework for making decisions, or in other words, making tradeoffs between different alternatives. In this framework, you follow step by step process. First, you identify what is the problem. Next, you make a table of objectives that you want to achieve and alternatives that you can think of. Fill in the expected consequences of each alternative in the cells. Based on the table, you make a tradeoff. During the process, URL should be kept in mind; uncertainties related to this problem, risk you can tolerate, and linked decisions you take following this decision \parencite{PrOACT}. 

The problem Narisa faces is how to get fabric lining for their dog cot stably. The objectives to achieve are durability, labor condition, environmental friendliness, supply stability, keeping production schedule, and price. The cause of the situation Narisa is currently in is the previous supplier went out of business. Even though not explicitly stated in the case study, supply stability is one of the important factors so her company doesn't have this kind of situation in the future. Alternatives are in-house production, Fab-Right, and All-Fabric. This isn't explicitly stated in the case either, but there's no reason K9 Dura-Corp cannot produce their material by themselves. So in-house production is one of the alternatives. The expected consequences for each alternative are described in the table below.

\begin{tabular}{|l|*{3}{p{1.2in}|}}
\hline
\textbf{}                         & \textbf{in-house production} & \textbf{Fab-Right}    & \textbf{All-Fabric}          \\ \hline
\textbf{durability}               & good                         & good                  & bad                          \\ \hline
\textbf{labor condition}          & good                         & exploit their workers & no problem is found          \\ \hline
\textbf{environmental friendliness}   & good                         & unknown               & they contaminate local water \\ \hline
\textbf{supply stability}         & good                         & unknown               & unknown                      \\ \hline
\textbf{keeping production schedule} & cannot start immediately     & OK                    & OK                           \\ \hline
\textbf{price}                    & needs feasibility study      & cheap                 & medium                       \\ \hline
\end{tabular}
  
There're several uncertanities regarding the matrix. First of all, at this moment, no one knows whether it's feasible for K9 Dura-Corp to produce materials by themselves. Maybe it's too costly, or even impossible for them to produce fabric. At the same time, you don't know how long it takes to find a supplier that meets Narisa's expectation, if there's any. As to environmental friendliness, Fab-Right may hide something as Narisa suspects, and All-Fabric may be punished by the government if current situation continues. K9 Dura-Corp may be critisized too for Fab-Right's labor exploitation, if they do business together. Even if K9 Dura-Corp chooses from Fab-Right and All-Fabric, there's risk that the chosen one goes out of business, just like the last supplier did. 

Risk we can tolerate in this situation would be supply instability. The first priority for K9 Dura-Corp is to keep their production schedule. It would be secondary importance whether the supplier can continuously suply their fabric. This, however, doesn't mean they can ignore labor exploitation and environmental pollution. Reputational risk is too huge to tolerate for ignoring such problems. Quality degradation is another risk they should avoid since it could damage customer loyalty and affect sales negatively. In either alternative, linked decisions have to be made to metigate these uncertainities and risk.

My recommendation for Narisa is she chooses Fab-Right for the next supplier, at least for now. First, since they cannot start in-house production now, they have to choose from Fab-Right or All-Fabric. Second, they should choose higher quolity fabric because they cannot tolerate quolity degradation, which could have negative impacts on sales, at this moment. Risks related to this decision are Fab-Right's working condition and envrionmental pollution. In order to handle them, K9 Dura-Corp should pay Fab-Right extra fees with which the supplier improves their working condition and environmental impacts of their manufacturing processes. Since their original price is cheaper than All-Fab, K9 Dura-Corp can still be profitable even after paying the fee. At the same time, K9 Dura-Corp should explore other possiblities such as in-house production, since relying on just one company makes their production unstable in case the company goes out of business.

604 words

\printbibliography
\end{document}
