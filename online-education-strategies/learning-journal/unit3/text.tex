\documentclass[paper=a4,12pt]{article}

\usepackage[american]{babel}
\usepackage{csquotes}
\usepackage[style=apa]{biblatex}
\usepackage{fontspec}
\usepackage[margin=1in]{geometry}
\usepackage{setspace} \doublespacing
\setmainfont{times new roman}
\addbibresource{export.bib}
\begin{document}
\usecounter
As to why netiquette is important, especially in the context of this university, respectful communication is one of the critical factors for effective learning. \parencite{Root1934}

Rapport, or trust between students and their instructors, is known to foster students' achievements \textcite{Root1934}. In this university, where a large part of education is conducted in the form of peer-to-peer, the rapport between students is essential to effective learning since each student is the other students' instructor at the same time. While it's critical to learning, trust between students is hard to create especially in online education, where direct face-to-face communication is missing. We need to pay extra care and attention to respectful communication more than normal verbal communication. This is one of the reasons why netiquette is important.

The consequences of bad netiquette would be less psychological safety and less constructive discussions. If I'm addressed with bad netiquette, I feel my perspective is unwanted and don't want to contribute to class discussions. Also, I don't feel safe to offer honest feedback to the offender. Both of those effects make class discussions inactive. I think others would feel the same way more or less. 

Netiquette affects peer assessment too. For example, a reviewer's rude attitude in peer assessment would make it emotionally difficult for their reviewee to accept their comment even if the criticism is reasonable. In that situation, constructive discussions are near impossible. On the other hand, if a reviewer is good-mannered, the handle for accepting criticism would be lowered.  

In the feedback I received this week, my peers appreciated my efforts in writing the post. That made me feel my point of view was wanted and I positively contributed to the class. I hope my feedback is taken in the same way, because I appreciate the points of view they bring to the class. 

\printbibliography
\end{document}
