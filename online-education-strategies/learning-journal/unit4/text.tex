\documentclass[paper=a4,12pt]{article}

\usepackage[american]{babel}
\usepackage{csquotes}
\usepackage[style=apa]{biblatex}
\usepackage{fontspec}
\usepackage[margin=1in]{geometry}
\usepackage{setspace} \doublespacing
\setmainfont{times new roman}
\addbibresource{export.bib}

\begin{document}

Critical thinking relates to peer assessment in the sense that the latter requires the former. Peer assessment is the process of evaluating the strength and weakness of each student's assertion and rating it. During the process, a reviewer has to check the validity of arguments, information, and data presented in their reviewee's work. This is exactly what \textcite{anonymous2015} describes as critical thinking.

The 5 tips presented in \textcite{ted-ed2016} can be applied to peer assessment with some modifications. The tips are 1. Formulate your question, 2. Gather your information, 3. Apply the information, 4. Consider the implications, and 5. Explore other points of view. In the case of our weekly writing assignment, the first part is often done by the instructor in the form of discussion prompts, and the second part can be considered as reading assignments and additional research conducted by each student. The parts after the 3rd phase can be used in the assessment. In the 3rd part, I as a reviewer analyze arguments and data presented in each student's work. I extract presented information into concepts that have clear patterns, examine assumptions and hidden value in the assertion, and check if the student's interpretation logically sounds. After analyzing the work, I consider the arguments' unintended impacts. Also, I explore if there're any other possible perspectives regarding the issue. In other words, I try to reframe the problem in a way different from the original work I'm reviewing. Following these steps will allow me to critically and systematically review other students' writing.

As to whether there is any surprising part in the assessment of writing assignments, the checkbox for a writing assignment is unchecked even after I have submitted it. After the grade assessment period is over the checkbox is checked. This is a bit confusing for the first time. If you know this, it won't be a problem the next time.

\printbibliography
\end{document}

