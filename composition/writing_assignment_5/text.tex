\documentclass[10pt,a4paper]{article}
\usepackage{natbib}
\begin{document}
  I chose "Solt" by Sherman because it was on the top of the list. The story is about an 18 years old intern in a newspaper company who was asked to write an obituary of Lois, a deceased obituary editor. He wrote her obituary and took over her job since then. As part of his job, he visited the house of Mona, an old lady who wanted to write her husband's obituary in the newspaper. She, however, had serious dementia and couldn't talk normally, so he left the house without receiving the obituary she said she was writing \citep{SaltShermanND}.

  I loved the story because it vividly contrasts different stages of line; youth, middle age, old age, and death. The intern is young and energetic and full of vigor. Death doesn't feel real to him yet. When Lois asked him how he wanted to remember him after his death, he said "I don’t want to think about that stuff. I’m eighteen" \citep{SaltShermanND}. Lois, who died at 45, wrote obituaries with wit and respect and accepted her own death as something that just happens like others she worked for. In her will, she asked the intern to use a normal template for obituary just like others she wrote obituaries for \citep{SaltShermanND}. Mona is very old and has dementia. She cannot recognize the death of her husband and her dog. She believes salt can bring back her dog and even her husband if she has enough of it. This story portrays life and death with humor and dignity.

One quote:
"How well can you mourn if you continually forget that the dead are dead?" \citep{SaltShermanND}

One paraphrase:
You need to recognize the deceased lives will not be back in order to fully mourn their death. \citep{SaltShermanND}

\bibliography{export}
\bibliographystyle{apalike-impure}
\end{document}
