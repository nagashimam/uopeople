\documentclass[10pt,a4paper]{article}
\usepackage{natbib}
\begin{document}
  The main argument of \citet{DayDreamingGaiman} is that fiction and library have a huge role and responsibility in literacy, which is fundation of our community. I do agree with his points and concerns, but I'd like to play the devil's advocate here. The way he argues has wider implications on discourses we have as a member of community we belong to. 
  One of the issues with his argument is he provided little evidence that is valid and relavant to his points. He had two points to support his argument: fictions are a gateway for other reading, and they cultivate imagination \citep{DayDreamingGaiman}. For the first point, there's no evidence in the essay that shows children read non-fictions after they have read fictions. That might be true for Gaiman, but not necessarily so for other kids. There're academic reports such as \citet{EngenderingCox} and \citet{IncorporatingCzerneda} that try to bridge science teaching and science fictions as he said, but few of them measure efficacy of their method in a verifiable way neither. It's untested whether those who read fictions eventually start reading others. 
  As to the second point, he uses example of people in GAFAM. According to him, the best and the brightest of the world there read science fictions when they were little. This doesn't necessarily mean science fictions foster their imagination and creativity. It could be the other way around. Science fictions requre readers the mental capacity to accept what is not real in the world we live in. Those in GAFAM might have been so smart even as a kid they could play with their mental capacity in the fictional world. If that's the case, they would have invented wonderful products even without science fictions. Correlation isn't causation. The two ideas his argument depends on don't have any basis other than his own belief and interpretation of the world. 
  What did he do instead of providing concreate supports for his point? He labeled opponents as "jailers" \citep[23]{DayDreamingGaiman}. Labeling opponents as public enenmy is common tactics seen elsewhere. Replace "jailers" with "racists" and "literacy" with "BLM". Or replace "jailers" with "murderers" and "literacy" with "right-to-life". How about "sexists" and "me too". You can basically use any arbitrary social movements to replace the worlds. We often see activists devide the world into "us" and "them". Does it lead to any meaningful discourse? Probably not. If we care about our own community, we should base our discussions on verifiable facts and logic instead of denouncing others. 
\bibliography{export}
\bibliographystyle{apalike}
\end{document}
